\chapter{Introduction \\
\small{\textit{-- Author Name}} 
\index{Chapter!introduction}
\index{introduction}
\label{Chapter::Introduction}}

% Add a section and label it so that we can reference it later
\section{My Section \label{Section::MySection}}

All projects should have a small introduction.  Here we provide some
example LaTeX commands.  The first one is an example on how to
introduce a PNG file as an image into the document, together with 
how to use a cite, such as this one \cite{GM1998}.

\begin{figure}
\centering
\scalebox{0.5}{\includegraphics{Figures/manAgileProcess.png}}
\caption{\label{Figure::manAgile} Figure of the continuous agile process.}
\end{figure}

% add a new page
\newpage

Hi there world!  Here is an example of a note\footnote{Here is a reference 
to Figure \ref{Figure::manAgile} and an indexed keyword\index{keyword}.}

\begin{enumerate}
\item One.
\item Two items.
\item Three or third item.
\end{enumerate}

And let's have a list, too:
\begin{itemize}
\item One item.
\item Two items.
\end{itemize}

\section{Electronics}

More details are available at \cite{circuitikz}

\tikz \draw (0,0) to[R=$R_1 \Omega$] (2,0);

\begin{circuitikz}
\draw (0,0) to[R=2$\Omega$, i=?, v=84V] (2,0) --
(2,2) to[V<=84V] (0,2)
 -- (0,0);
\end{circuitikz}


Draw some op-amps:

\begin{circuitikz}[]
	\draw (0,0) node[above]{$v_i$} to[short, o-] ++(1,0)
	node[op amp, noinv input up, anchor=+](OA){\texttt{OA1}}
	;
\end{circuitikz}

\begin{circuitikz}[scale=0.8, transform shape]
 \draw (0,0) node[above]{$v_i$} to[short, o-] ++(1,0)
 node[op amp, noinv input up, anchor=+](OA){\texttt{OA1}}
 (OA.-) -- ++(0,-1) coordinate(FB)
 to[R=$R_1$] ++(0,-2) node[ground]{}
 (FB) to[R=$R_2$, *-] (FB -| OA.out) -- (OA.out)
 to [short, *-o] ++(1,0) node[above]{$v_o$}
 ;
\end{circuitikz}

\newpage
Draw some digital stuff:

\ctikzset{
 logic ports=ieee,
 logic ports/scale=0.7,
}
\tikzset{sr-ff/.style={flipflop, flipflop def={
 t1=S, t2=CP, t3=R, t4={\ctikztextnot{Q}},
 t6=Q, nd=1}},
}
\begin{circuitikz}[]
 \draw  (0,0) node[sr-ff](FF){} (FF.bup) node[above]{SR-FF};
 \draw  (FF.pin 1) -- ++(-1,0) node[and port, anchor=out](AND1){}
	    (FF.pin 3) -- ++(-1,0) node[and port, anchor=out](AND2){};
\end{circuitikz}